\subsection{loadCurves}

\subsubsection{create\_nasa\_curves}

\begin{Shaded}
\begin{Highlighting}[]
\NormalTok{create_nasa_curves()}
\end{Highlighting}
\end{Shaded}

Creates a matrix of all NASA battery curves, and its labels

Creates each curve as a row in in a large X matrix. Each curve is
voltage interpolated to be in 128 steps as a function of the power
delivered from 2\% to 98\%.

\subsubsection{load\_nasa\_curves}

\begin{Shaded}
\begin{Highlighting}[]
\NormalTok{load_nasa_curves(to_load)}
\end{Highlighting}
\end{Shaded}

Loads the curves cached to disk

\textbf{Returns}:

\begin{itemize}
\tightlist
\item
  \texttt{X} \emph{ndarray} - Matrix of curves
\item
  \texttt{p} \emph{ndarray} - The discharge percentage that X rows are a
  function of
\item
  \texttt{labels} \emph{ndarray} - Whuch battery each curve is
\item
  \texttt{capacity} \emph{ndarray} - The capacity of each battery during
  each curve
\end{itemize}

\subsection{main}

\subsubsection{plot\_aging}

\begin{Shaded}
\begin{Highlighting}[]
\NormalTok{plot_aging(p, X, labels, capacity)}
\end{Highlighting}
\end{Shaded}

Plot examples of new and old battery curves.

\textbf{Arguments}:

\begin{itemize}
\tightlist
\item
  \texttt{X} \emph{ndarray} - Matrix of curves
\item
  \texttt{p} \emph{ndarray} - The discharge percentage that X rows are a
  function of
\item
  \texttt{labels} \emph{ndarray} - Whuch battery each curve is
\item
  \texttt{capacity} \emph{ndarray} - The capacity of each battery during
  each curve
\end{itemize}

\subsection{svd}

\subsubsection{plot\_mode\_fraction}

\begin{Shaded}
\begin{Highlighting}[]
\NormalTok{plot_mode_fraction(s)}
\end{Highlighting}
\end{Shaded}

Plots the fraction of power represented with n modes.

\textbf{Arguments}:

\begin{itemize}
\tightlist
\item
  \texttt{s} \emph{array-like} - 1D arrray of the variances of the
  principal components.
\end{itemize}

\subsubsection{plot\_n\_modes}

\begin{Shaded}
\begin{Highlighting}[]
\NormalTok{plot_n_modes(p, X, V, n)}
\end{Highlighting}
\end{Shaded}

Shows the numbers represented with the selected number of SVD modes.

\textbf{Arguments}:

\begin{itemize}
\tightlist
\item
  \texttt{p} \emph{array\_like} - The discharge percentage that X rows
  are a function of.
\item
  \texttt{X} \emph{array\_like} - Data matrix with rows of images.
\item
  \texttt{V} \emph{array\_like} - Matrix with mode vectors as columns.
\item
  \texttt{n} \emph{int} - Number of modes to use in the representation.
\end{itemize}

\subsubsection{plot\_mode\_vs\_life}

\begin{Shaded}
\begin{Highlighting}[]
\NormalTok{plot_mode_vs_life(capacity, X, V, n)}
\end{Highlighting}
\end{Shaded}

Plots how the capacity changes for values of each SVD mode.

\textbf{Arguments}:

\begin{itemize}
\tightlist
\item
  \texttt{capacity} \emph{ndarray} - The capacity of each battery during
  each curve.
\item
  \texttt{X} \emph{ndarray} - Matrix of curves.
\item
  \texttt{V} \emph{array\_like} - Matrix with mode vectors as columns.
\item
  \texttt{n} \emph{int} - Number of modes to plot.
\end{itemize}

\subsection{predict}

\subsubsection{find\_weights}

\begin{Shaded}
\begin{Highlighting}[]
\NormalTok{find_weights(X, b)}
\end{Highlighting}
\end{Shaded}

Calculate the weights of the linear regression.

\textbf{Arguments}:

\begin{itemize}
\tightlist
\item
  \texttt{X} \emph{array-like} - Matrix with rows of curves.
\item
  \texttt{b} \emph{array-like} - The capacity for each curve.
\end{itemize}

\textbf{Returns}:

\begin{itemize}
\tightlist
\item
  \texttt{ndarray} - Weights for the model.
\end{itemize}

\subsection{predict}

\begin{Shaded}
\begin{Highlighting}[]
\NormalTok{predict(X, w)}
\end{Highlighting}
\end{Shaded}

Using previously calculated weights, predict the capacity of each curve.

\textbf{Arguments}:

\begin{itemize}
\tightlist
\item
  \texttt{X} \emph{array-like} - Matrix with rows of curves.
\item
  \texttt{w} \emph{array-like} - Weights for the model.
\end{itemize}

\textbf{Returns}:

\begin{itemize}
\tightlist
\item
  \texttt{ndarray} - Predicted Capacities
\end{itemize}

\subsection{accuracy\_dist}

\begin{Shaded}
\begin{Highlighting}[]
\NormalTok{accuracy_dist(b_real, b_model, path_out)}
\end{Highlighting}
\end{Shaded}

Plots the error for two sets of capacties

\textbf{Arguments}:

\begin{itemize}
\tightlist
\item
  \texttt{b\_real} \emph{array-like} - The capacity actually measured.
\item
  \texttt{b\_model} \emph{array-like} - Predicted capacities.
\end{itemize}
