\subsection{spectrum}

\subsubsection{plot\_spectrum}

\begin{Shaded}
\begin{Highlighting}[]
\NormalTok{plot_spectrum(f, t, S, name, freq_range}\OperatorTok{=}\NormalTok{(}\DecValTok{50}\NormalTok{, }\DecValTok{2000}\NormalTok{))}
\end{Highlighting}
\end{Shaded}

Takes a provided rectangular array of spectrum vs time and creates a
meshplot with the markings for notes of the equal tempered scale.

\textbf{Arguments}:

\begin{itemize}
\tightlist
\item
  \texttt{f} \emph{array\_like} - A 1D array of frequencies for the
  spectrum in P
\item
  \texttt{t} \emph{array\_like} - Time samples of the spectrum.
\item
  \texttt{S} \emph{array\_like} - An array of spectra at different
  times. Should be of shape (len(t), len(f)).
\item
  \texttt{name} \emph{str} - The output file same. The plot will be
  saved in ``HW2/figures/name.png''.
\item
  \texttt{freq\_range} \emph{tuple} - The ends of the frequency range to
  plot. Defaults to (50 Hz, 2000 Hz).
\end{itemize}

\subsubsection{get\_spectrum}

\begin{Shaded}
\begin{Highlighting}[]
\NormalTok{get_spectrum(data, rate, width)}
\end{Highlighting}
\end{Shaded}

Produces the Gabor transform of the data provided using a gaussian
window with the width provided. The transform is given in timesteps of
1/4 the width.

\textbf{Arguments}:

\begin{itemize}
\tightlist
\item
  \texttt{data} \emph{array\_like} - The time series to be transformed.
\item
  \texttt{rate} \emph{float} - The sampling rate (\#/s) used for data.
\item
  \texttt{width} \emph{float} - The width of the gaussian window in
  seconds.
\end{itemize}

\textbf{Returns}:

\begin{itemize}
\tightlist
\item
  \texttt{f} \emph{ndarray} - Array of sample frequencies.
\item
  \texttt{t} \emph{ndarray} - Array of sample times.
\item
  \texttt{S} \emph{ndarray} - The Gabor transform of data.
\end{itemize}

\subsubsection{low\_pass}

\begin{Shaded}
\begin{Highlighting}[]
\NormalTok{low_pass(data, rate, cutoff)}
\end{Highlighting}
\end{Shaded}

Applies a 4th order low pass filter to a time series.

\textbf{Arguments}:

\begin{itemize}
\tightlist
\item
  \texttt{data} \emph{array\_like} - The time series to be filtered.
\item
  \texttt{rate} \emph{float} - The sample rate of data.
\item
  \texttt{cutoff} \emph{float} - Cutoff frequency for the filter in Hz.
\end{itemize}

\subsubsection{run\_gnr}

\begin{Shaded}
\begin{Highlighting}[]
\NormalTok{run_gnr()}
\end{Highlighting}
\end{Shaded}

Generates the spectrum of the Sweet Child O' Mine clip.

\subsubsection{run\_floyd\_guitar}

\begin{Shaded}
\begin{Highlighting}[]
\NormalTok{run_floyd_guitar()}
\end{Highlighting}
\end{Shaded}

Generates the spectrum of the guitar in the Comfortably Numb clip.

\subsubsection{run\_floyd\_bass}

\begin{Shaded}
\begin{Highlighting}[]
\NormalTok{run_floyd_bass()}
\end{Highlighting}
\end{Shaded}

Generates the spectrum of the bass in the Comfortably Numb clip, and
produces a filtered clip of just the bass.
