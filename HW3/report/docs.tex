
\subsection{Video Objects}

\begin{Shaded}
\begin{Highlighting}[]
\KeywordTok{class}\NormalTok{ Video()}
\end{Highlighting}
\end{Shaded}

Parameters of to load a single video matrix.

This is to load the video form the matlab matrix file and then crop it
appropriately.

\textbf{Attributes}:

\begin{itemize}
\tightlist
\item
  \texttt{filename} \emph{str} - The path to the matlab matrix file for
  the video.
\item
  \texttt{start} \emph{int} - The frame to start loading from.
\item
  \texttt{end} \emph{int} - Load frames before this number.
\item
  \texttt{left} \emph{int} - The left edge to crop from.
\item
  \texttt{right} \emph{int} - The right edge to crop from.
\item
  \texttt{top} \emph{int} - The top edge to crop from.
\item
  \texttt{bottom} \emph{int} - The bottom edge to crop from.
\end{itemize}

\subsubsection{\_\_init\_\_}

\begin{Shaded}
\begin{Highlighting}[]
 \OperatorTok{|} \FunctionTok{__init__}\NormalTok{(filename, start, end, left, right, top, bottom)}
\end{Highlighting}
\end{Shaded}

Initialize the Video class.

args: filename (str): The path to the matlab matrix file for the video.
start (int): The frame to start loading from. end (int): Load frames
before this number. left (int): The left edge to crop from. right (int):
The right edge to crop from. top (int): The top edge to crop from.
bottom (int): The bottom edge to crop from.

\subsubsection{read}

\begin{Shaded}
\begin{Highlighting}[]
 \OperatorTok{|}\NormalTok{ read()}
\end{Highlighting}
\end{Shaded}

Retrieve the matrix of the video, cropped as specified.

\textbf{Returns}:

\begin{itemize}
\tightlist
\item
  \texttt{ndarray} - The matrix of the video. The shape is (vertical
  pixels, horizontal pixels, frames).
\end{itemize}

\subsubsection{from\_text}

\begin{Shaded}
\begin{Highlighting}[]
 \OperatorTok{|} \OperatorTok{@}\NormalTok{staticmethod}
 \OperatorTok{|}\NormalTok{ from_text(text)}
\end{Highlighting}
\end{Shaded}

Creates a video class from the line of a CSV

\textbf{Arguments}:

\begin{itemize}
\tightlist
\item
  \texttt{text} \emph{str} - Line from a csv ``path, start, end, left,
  right, top, bottom''
\end{itemize}

\textbf{Returns}:

\begin{itemize}
\tightlist
\item
  \texttt{Video} - A \texttt{Video} class for the line provided.
\end{itemize}

\subsection{make\_vid\_list}

\begin{Shaded}
\begin{Highlighting}[]
\NormalTok{make_vid_list()}
\end{Highlighting}
\end{Shaded}

Create lists of \texttt{Video} for all 4 tests.txt

Creates the videos with the properties defined in the vid\_props files.

\textbf{Returns}:

\begin{itemize}
\tightlist
\item
  \texttt{list} - List of each test's list of \texttt{Video} objects.
\end{itemize}

\subsection{read\_meas\_matrix}

\begin{Shaded}
\begin{Highlighting}[]
\NormalTok{read_meas_matrix(vid_list)}
\end{Highlighting}
\end{Shaded}

Loads the matrix with rows being the separate time measurements.

This is the X matrix expected for PCA.

\textbf{Arguments}:

\begin{itemize}
\tightlist
\item
  \texttt{vid\_list} \emph{list} - A list of \texttt{Video} class
  objects to load the measurements from. All of the should be the same
  length.
\end{itemize}

\textbf{Returns}:

\begin{itemize}
\tightlist
\item
  \texttt{ndarray} - The measurements matrix X for PCA. Shape (Number of
  Pixels, Number of Frames)
\end{itemize}

\subsection{pca}

\begin{Shaded}
\begin{Highlighting}[]
\NormalTok{pca(M)}
\end{Highlighting}
\end{Shaded}

Preforms PCA on the matrix provided.

\textbf{Arguments}:

\begin{itemize}
\tightlist
\item
  \texttt{M} \emph{array-like} - The matrix to preform PCA on.
\end{itemize}

\textbf{Returns}:

\begin{itemize}
\tightlist
\item
  \texttt{ndarray} - The variances of the principal components.
\item
  \texttt{ndarray} - U\_T matrix to project M into principal components.
\end{itemize}

\subsection{plot\_dominant\_mode}

\begin{Shaded}
\begin{Highlighting}[]
\NormalTok{plot_dominant_mode(s, U_T, X, fig_path)}
\end{Highlighting}
\end{Shaded}

Plots the variances and the fist 4 PCA modes.

\textbf{Arguments}:

\begin{itemize}
\tightlist
\item
  \texttt{s} \emph{array-like} - 1D arrray of the variances of the
  principal components.
\item
  \texttt{U\_T} \emph{array-like} - The matrix to transform X into the
  principal components.
\item
  \texttt{X} \emph{array-like} - The measurement matrix
\item
  \texttt{fig\_path} \emph{str} - The path to save the plot.
\item
  \texttt{comps} \emph{int} - Number of modes to plot
\end{itemize}
