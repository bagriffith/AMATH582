\subsection{LDA Objects}

\begin{Shaded}
\begin{Highlighting}[]
\KeywordTok{class}\NormalTok{ LDA()}
\end{Highlighting}
\end{Shaded}

Linear Discrimination Analysis model for classifying in up to 3 groups.

\subsubsection{fit}

\begin{Shaded}
\begin{Highlighting}[]
 \OperatorTok{|}\NormalTok{ fit(X, y)}
\end{Highlighting}
\end{Shaded}

Trains the model.

\textbf{Arguments}:

\begin{itemize}
\tightlist
\item
  \texttt{X} \emph{array-like} - Contains rows of the training data
  examples.
\item
  \texttt{y} \emph{array-like} - Contains labels for the training data
  rows.
\end{itemize}

\subsubsection{predict}

\begin{Shaded}
\begin{Highlighting}[]
 \OperatorTok{|}\NormalTok{ predict(X)}
\end{Highlighting}
\end{Shaded}

Predicts the category of rows of X.

\textbf{Arguments}:

\begin{itemize}
\tightlist
\item
  \texttt{X} \emph{array-like} - Contains rows of the data to
  categorize.
\end{itemize}

\subsection{evaluation}

\subsubsection{NaiveClassifier Objects}

\begin{Shaded}
\begin{Highlighting}[]
\KeywordTok{class}\NormalTok{ NaiveClassifier()}
\end{Highlighting}
\end{Shaded}

A model classifier that randomly guesses a digit.

This was created as a simple test article to make sure the
number\_confusion code worked indpendent of any model used.

\subsubsection{number\_confusion}

\begin{Shaded}
\begin{Highlighting}[]
\NormalTok{number_confusion(model, train_n, V)}
\end{Highlighting}
\end{Shaded}

Plots a how the model preforms at distinguising pairs of digits.

\textbf{Arguments}:

\begin{itemize}
\tightlist
\item
  \texttt{model} - The model class. Should have functions fit(X, y) that
  trains the model to identify labels y using data X and predict(X) that
  will label data in the matrix X.
\item
  \texttt{train\_n} \emph{int} - The number of example digits to train
  on.
\item
  \texttt{V} \emph{array-like} - A matrix to transform the data into the
  basis for predictions.
\end{itemize}

\subsubsection{full\_classification}

\begin{Shaded}
\begin{Highlighting}[]
\NormalTok{full_classification(model, train_n, V)}
\end{Highlighting}
\end{Shaded}

Plots a how the model preforms at identifying digits.

\textbf{Arguments}:

\begin{itemize}
\tightlist
\item
  \texttt{model} - The model class. Should have functions fit(X, y) that
  trains the model to identify labels y using data X and predict(X) that
  will label data in the matrix X.
\item
  \texttt{train\_n} \emph{int} - The number of example digits to train
  on.
\item
  \texttt{V} \emph{array-like} - A matrix to transform the data into the
  basis for predictions.
\end{itemize}

\subsubsection{digit\_performance}

\begin{Shaded}
\begin{Highlighting}[]
\NormalTok{digit_performance(model, train_n, V, digits)}
\end{Highlighting}
\end{Shaded}

Plots a how the model preforms at identifying digits.

\textbf{Arguments}:

\begin{itemize}
\tightlist
\item
  \texttt{model} - The model class. Should have functions fit(X, y) that
  trains the model to identify labels y using data X and predict(X) that
  will label data in the matrix X.
\item
  \texttt{train\_n} \emph{int} - The number of example digits to train
  on.
\item
  \texttt{V} \emph{array-like} - A matrix to transform the data into the
  basis for predictions.
\item
  \texttt{digits} \emph{list} - List of digits to test on
\end{itemize}

\subsection{main}

\subsubsection{run\_analysis}

\begin{Shaded}
\begin{Highlighting}[]
\NormalTok{run_analysis()}
\end{Highlighting}
\end{Shaded}

Runs the full analysis for the MNIST handwritting project

\subsection{svd}

\subsubsection{plot\_mode\_proj}

\begin{Shaded}
\begin{Highlighting}[]
\NormalTok{plot_mode_proj(X, V, labels, modes)}
\end{Highlighting}
\end{Shaded}

Creates a 3D projection of X into the 3 selected SVD modes

\textbf{Arguments}:

\begin{itemize}
\tightlist
\item
  \texttt{X} \emph{array\_like} - Data matrix with rows of images
\item
  \texttt{V} \emph{array\_like} - Matrix with mode vectors as columns
\item
  \texttt{modes} \emph{list} - List of 3 mode indexes to project on
\end{itemize}

\subsubsection{plot\_n\_modes}

\begin{Shaded}
\begin{Highlighting}[]
\NormalTok{plot_n_modes(X, V, n)}
\end{Highlighting}
\end{Shaded}

Shows the numbers represented with the selected number of SVD modes

\textbf{Arguments}:

\begin{itemize}
\tightlist
\item
  \texttt{X} \emph{array\_like} - Data matrix with rows of images
\item
  \texttt{V} \emph{array\_like} - Matrix with mode vectors as columns
\item
  \texttt{n} \emph{int} - Number of modes to use in the representation
\end{itemize}

\subsubsection{plot\_svd\_spectrum}

\begin{Shaded}
\begin{Highlighting}[]
\NormalTok{plot_svd_spectrum(X, V)}
\end{Highlighting}
\end{Shaded}

Plots the svd spectrum of 4 random images.

\textbf{Arguments}:

\begin{itemize}
\tightlist
\item
  \texttt{X} \emph{array\_like} - Data matrix with rows of images
\item
  \texttt{V} \emph{array\_like} - Matrix with mode vectors as columns
\end{itemize}

\subsubsection{plot\_mode\_fraction}

\begin{Shaded}
\begin{Highlighting}[]
\NormalTok{plot_mode_fraction(s)}
\end{Highlighting}
\end{Shaded}

Plots the fraction of power represented with n modes

\textbf{Arguments}:

\begin{itemize}
\tightlist
\item
  \texttt{s} \emph{array-like} - 1D arrray of the variances of the
  principal components.
\end{itemize}

\subsection{loadmnist}

\subsubsection{load\_data}

\begin{Shaded}
\begin{Highlighting}[]
\NormalTok{load_data(numbers}\OperatorTok{=}\VariableTok{None}\NormalTok{, size}\OperatorTok{=}\VariableTok{None}\NormalTok{)}
\end{Highlighting}
\end{Shaded}

Loads a matrix of selected numbers.

Creates a matrix of shape (nsamples, npixels) where nsamples is the
number of occurrences of the selected numbers.

\textbf{Arguments}:

\begin{itemize}
\tightlist
\item
  \texttt{numbers} \emph{list} - A list of digits to load. If None or
  empty, all digits will by loaded. Defaults to None.
\item
  \texttt{size} \emph{int} - The max number of images to load. If None,
  all images will be loaded. Defaults to None.
\end{itemize}

\textbf{Returns}:

\begin{itemize}
\tightlist
\item
  \texttt{np.int32} - Matrix with rows of images
\item
  \texttt{np.int8} - Array of digit labels for rows of images
\end{itemize}
